\documentclass[aspectratio=169]{beamer}
\usepackage[utf8]{luainputenc}
\usepackage{amssymb,amsmath}
\usepackage{graphicx}
\usepackage{tikz}
\usepackage{beamercolorthemetud}
\usepackage[backend=biber]{biblatex}
\usepackage{caption}
\captionsetup[figure]{font=scriptsize}
\usepackage{hyperref}

\usepackage{subcaption}
\usepackage[absolute,overlay]{textpos}
  \setlength{\TPHorizModule}{1mm}
  \setlength{\TPVertModule}{1mm}




\addbibresource{bibl.bib}
\setbeamertemplate{bibliography item}{\insertbiblabel}
\usepackage[english]{babel}
\usetheme[]{tud}
\setbeamercolor{background canvas}{bg=}
\setbeamerfont{frametitle}{size=\Large}
\input{macros}



\title{MeetForSport: Adaptation Concept}
\author{Mattis Lahr, Felix Fischer}
\date{10.12.2021}

\einrichtung{\hspace{-1pt}Institute of Systems Architecture}
\datecity{Dresden}




\AtBeginSection[]{\partpage{\usebeamertemplate***{part page}}}
\begin{document}
\maketitle



\begin{frame}
    \frametitle{Table of Contents}
    \tableofcontents
\end{frame}



\section{Hook}
\begin{frame}
\frametitle{App Idea}
This app will allow users to join group activities (i.e. football) or join ongoing events.
Targeting mostly active persons, this small social network will allow users  to find new friends/persons with the same interests and therefore allow these people to become more active.
\end{frame}

\section{Need}
	\begin{frame}
		\frametitle{Use Cases}
	\end{frame}

	\begin{frame}
		\frametitle{Challenges and need for Adaptation}
	\end{frame}

\section{Approach}

	\begin{frame}
		\frametitle{Main Features}
	\end{frame}

	\begin{frame}
		\frametitle{Adaptation Concepts}
	\end{frame}

	\begin{frame}
		\frametitle{Architecture}
	\end{frame}

\section{Benefits}

	\begin{frame}
		\frametitle{Benefits}
	\end{frame}

\section{Team}

	\begin{frame}
		\frametitle{Team}
	\end{frame}

\end{document}