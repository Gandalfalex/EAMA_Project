\documentclass[aspectratio=169]{beamer}
\usepackage[utf8]{luainputenc}
\usepackage{amssymb,amsmath}
\usepackage{graphicx}
\usepackage{tikz}
\usepackage{beamercolorthemetud}
\usepackage[backend=biber]{biblatex}
\usepackage{caption}
\captionsetup[figure]{font=scriptsize}

\usepackage{subcaption}
\usepackage[absolute,overlay]{textpos}
  \setlength{\TPHorizModule}{1mm}
  \setlength{\TPVertModule}{1mm}




\addbibresource{bibl.bib}
\setbeamertemplate{bibliography item}{\insertbiblabel}
\usepackage[english]{babel}
\usetheme[]{tud}
\setbeamercolor{background canvas}{bg=}
\setbeamerfont{frametitle}{size=\Large}
\input{macros}



\title{App Idea Presentation: MeetForSport}
\author{Mattis Lahr, Felix Fischer}
\date{19.11.2021}

\einrichtung{\hspace{-1pt}Institute of Systems Architecture}
\datecity{Dresden}




\AtBeginSection[]{\partpage{\usebeamertemplate***{part page}}}
\begin{document}
\maketitle



\begin{frame}
    \frametitle{Table of Contents}
    \tableofcontents
\end{frame}





\section{Scenario}
\begin{frame}   
\frametitle{Our Scenario}
Imagine:
    \begin{itemize}	
	 	\item You are an active person
	 	\item You like to play team sports but don’t know how to find enough other people interested in the sport of your choice to actually built big enough teams OR
	 	\item You just want to find other people who are interested in the same sports as you to do them together
    \end{itemize}
\end{frame}


\section{Problem}
\begin{frame}   
	While there are many (fitness) Apps out on the market, none really
	addresses the problem of lacking people to play with. Joining an active
	community is still part of ones social circle, whereas every other aspect of
	our live can be enhanced by multimedia devices.
	\end{frame}


\section{Our Solution}
\begin{frame}   
\frametitle{What is our idea?}
Our Solution is rather simple and draws its inspiration from other networking apps.
Our App would try to solve these problems. It is supposed to become an App that
    \begin{itemize}	
		\item enables you to find people interested in the same sport
	 	\item suggests places for your activity 
	 	\item lets you set up meeting times and points
	 	\item builds a community around your activities 
    \end{itemize}
\end{frame}

\begin{frame}   
\frametitle{Target Group}
The App will target persons of nearly all ages who are interested in physical group
activities
\end{frame}

\begin{frame}   
	\frametitle{Personas}{\textbf{Fred Flintstone Part 1:}}

	This is Fred Flintstone. He is 23 years old and studies Physics in his 7th semester. 
He's from Dresden and currently resids in the studentdorm. 
When not studying, he takes up his old hobby of playing football, that he started in preschool. If the weather is bad, he fires up his playstation to join his friends in some multiplayer game.
On his weekends, Fred likes to meet up with people in pubs or clubs.

However, one thing really annoys him... unreliable friends. Setting up sport events with them proves difficult in many ways.
One of them will surely have forgotten the date, one of them cancels last second and others allways run late.
So while he enjoys playing football, setting up the event is quite a challenge.
\end{frame}

\begin{frame}   
	\frametitle{Personas}{\textbf{Fred Flintstone Part 2:}}

\begin{itemize}
	\item personal information:
	\begin{itemize}
		\item Age: 23
		\item Profession: physics stundent (7th semester)
		\item Living condition: alone, stundentdorm
		\item Interests: football, meeting friends, playing (computer) games, parties
	\end{itemize}
	\item personal trades:
	\begin{itemize}
		\item emphatic
		\item organized
		\item spontaneous
		\item tech-savvy
\end{itemize}
\end{itemize}

\end{frame}

\begin{frame}   
	\frametitle{Personas}{\textbf{Martina Ödegaard Part 1:}}

	This is Martina Ödegaard. She is a 36 year old car mechanic. She is originally from Frankfurt(Oder) where she also learned her profession. She moved to Dresden 12 years ago, after moving in with her partner, who was already living there. 
\end{frame}

\begin{frame}   
	\frametitle{Personas}{\textbf{Martina Ödegaard Part 2:}}

\begin{itemize}
	\item personal information:
	\begin{itemize}
		\item Age: 36
		\item Profession: Car mechanic
		\item Living condition: married, living in an appartment with her partner (37) and her two kids (5 and 8)
		\item Interests: Reading, Running, Meeting with friends
	\end{itemize}
	\item personal trades:
	\begin{itemize}
		\item
\end{itemize}
\end{itemize}

\end{frame}


\begin{frame}
\frametitle{Key Features}
For the time being, we think about building an Android App based on the
Server-Client Model. The App includes:

\begin{itemize}
	\item an interactive map showing sport fields and other similiar locations
	\item the possibility to create group activities
	\item customizable and adaptable user profiles 
\end{itemize}
In summary, the app helps people in neighbouring areas who don’t necessarily
know each other to organise joint sporting activities. This also makes it a good tool
for people who are interested in the same sports to get to know each other.
\end{frame}


\section{Mockups}


\section{Challenges}
\begin{frame}
	\begin{itemize}
		\item build/find an efficient way to find locations
		\item assign each sport a set of requirements (e.g. important features the location needs)
		\item build a user database
		\item build an organizer 
		\item implement interaction possibilities
	\end{itemize}
\end{frame}

\end{document}